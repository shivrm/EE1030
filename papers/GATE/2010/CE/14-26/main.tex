%iffalse
\documentclass[journal]{IEEEtran}
\usepackage[a5paper, margin=10mm]{geometry}
%\usepackage{lmodern} % Ensure lmodern is loaded for pdflatex
\usepackage{tfrupee} % Include tfrupee package


\setlength{\headheight}{1cm} % Set the height of the header box
\setlength{\headsep}{0mm}     % Set the distance between the header box and the top of the text


%\usepackage[a5paper, top=10mm, bottom=10mm, left=10mm, right=10mm]{geometry}

%
\usepackage{gvv-book}
\usepackage{gvv}
\setlength{\intextsep}{10pt} % Space between text and floats

\makeindex

\begin{document}
\bibliographystyle{IEEEtran}
\onecolumn
%endif

\title{2010 CE 14-26}
\author{AI24BTECH11031 - Shivram S}
\maketitle
\bigskip

\renewcommand{\thefigure}{\theenumi}
\renewcommand{\thetable}{\theenumi}

\begin{enumerate}
    \item The $e$-$\log p$ curve shown in the figure is representative of 
   
        \begin{center}
        \begin{tikzpicture}
            \draw[->] (0,0) -- (5,0) node[midway, below] {$\log p$};
            \draw[->] (0,0) -- (0,4) node[midway, above, rotate=90] {Void ratio, $e$};
            \draw (0.5,3.5) to[out=0,in=120] (4,1.5);
        \end{tikzpicture}
        \end{center}

        \begin{multicols}{2}
            \begin{enumerate}
                \item Normally consolidated clay
                \item Over consolidated clay
                \item Under consolidated clay
                \item Normally consolidated clayey sand
            \end{enumerate}
        \end{multicols}

    \item If $\sigma_h, \sigma_v, \sigma_h'$, and $\sigma_v'$ represent the total horizontal stress,
    total vertical stress, effective horizontal stress, and effective vertical stress on a soil
    element, respectively, the coefficient of earth pressure at rest is given by  

        \begin{multicols}{4}
            \begin{enumerate}
                \item $\frac{\sigma_h}{\sigma_v}$
                \item $\frac{\sigma_h'}{\sigma_v'}$
                \item $\frac{\sigma_v}{\sigma_h}$
                \item $\frac{\sigma_v'}{\sigma_h'}$
            \end{enumerate}
        \end{multicols}

    \item A mild-sloped channel is followed by a steep-sloped channel. The profiles of gradually
    varied flow in the channel are  

        \begin{multicols}{4}
            \begin{enumerate}
                \item $M_3, S_2$
                \item $M_3, S_3$
                \item $M_2, S_1$
                \item $M_2, S_2$
            \end{enumerate}
        \end{multicols}

    \item The flow in a rectangular channel is subcritical. If the width of the channel is reduced
    at a certain section, the water surface under no-choke condition will  

        \begin{multicols}{2}
            \begin{enumerate}
                \item drop at a downstream section
                \item rise at a downstream section
                \item rise at an upstream section
                \item not undergo any change
            \end{enumerate}
        \end{multicols}
       
    \item The correct match of Group-I with Group-II is
        \begin{multicols}{2}
            Group-I
            \begin{enumerate}[label={\Alph*.}]
                \setcounter{enumii}{15}
                \item Evapotranspiration
                \item Infiltration
                \item Synthetic unit hydrograph
                \item Channel Routing
            \end{enumerate}
            \columnbreak
            Group-II
            \begin{enumerate}[label={\arabic*.}]
                \item Penman method
                \item Snyder's method
                \item Muskingum method
                \item Horton's method
            \end{enumerate}
        \end{multicols}

        \begin{multicols}{2}
            \begin{enumerate}
                \item P-1, Q-3, R-4, S-2
                \item P-1, Q-4, R-2, S-3
                \item P-3, Q-4, R-1, S-2
                \item P-4, Q-2, R-1, S-3
            \end{enumerate}
        \end{multicols}

    \item Group-I gives a list of devices and Group-II gives the list of uses.
        \begin{multicols}{2}
            Group-I
            \begin{enumerate}[label={\Alph*.}]
                \setcounter{enumii}{15}
                \item Pitot tube
                \item Manometer
                \item Venturimeter
                \item Anemometer
            \end{enumerate}
            \columnbreak
            Group-II
            \begin{enumerate}[label={\arabic*.}]
                \item measuring pressure in a pipe
                \item measuring velocity of flow in a pipe
                \item measuring air and gas velocity
                \item measuring discharge in a pipe
            \end{enumerate}
        \end{multicols}
        The correct match of Group-I with Group-II is
        
        \begin{multicols}{2}
            \begin{enumerate}
                \item P-1, Q-2, R-4, S-3
                \item P-2, Q-1, R-3, S-4
                \item P-2, Q-1, R-4, S-3
                \item P-4, Q-1, R-3, S-2
            \end{enumerate}
        \end{multicols}

    \item A coastal city produces municipal solid waste (MSW) with high moisture
    content, high organic materials, low calorific value, and low inorganic materials.
    The most effective and sustainable option for MSW management in that city is  

        \begin{multicols}{4}
            \begin{enumerate}
                \item Composting
                \item Dumping in sea
                \item Incineration
                \item Landfill
            \end{enumerate}
        \end{multicols}

    \item According to the Noise Pollution (Regulation and Control) Rules, 2000,
    of the Ministry of Environment and Forests, India, the day time and night time
    noise level limits in ambient air for residential areas expressed in dB(A) Leq are
        \begin{multicols}{4}
            \begin{enumerate}
                \item 50 and 40
                \item 55 and 45
                \item 65 and 55
                \item 75 and 70
            \end{enumerate}
        \end{multicols}

        \item An air parcel having $40\degree C$ temperature moves from ground level to
        500 m elevation in dry air following the "adiabatic lapse rate". The resulting  
        temperature of air parcel at 500 m elevation will be
        \begin{multicols}{4}
            \begin{enumerate}
                \item $35^{\circ}C$
                \item $38^{\circ}C$
                \item $41^{\circ}C$
                \item $44^{\circ}C$
            \end{enumerate}
        \end{multicols}

    \item Aggregate impact value indicates the following property of aggregates
        \begin{multicols}{4}
            \begin{enumerate}
                \item Durability
                \item Toughness
                \item Hardness
                \item Strength
            \end{enumerate}
        \end{multicols}

    \item As per IRC: 67-2001, a traffic sign indicating the Speed Limit on a road should be of
            \begin{enumerate}
                \item Circular Shape with White Background and Red Border
                \item Triangular Shape with White Background and Red Border
                \item Triangular Shape with Red Background and White Border
                \item Circular Shape with Red Background and White Border
            \end{enumerate}

    \item The local mean time at a place located in longitude $90\degree 40'$ E when the standard
    time is 6 hours and 30 minutes and the standard meridian is $82\degree 30'$ E is
        \begin{multicols}{2}
            \begin{enumerate}
                \item 5 hours, 2 minutes and 40 seconds
                \item 5 hours, 57 minutes and 20 seconds
                \item 6 hours and 30 minutes
                \item 7 hours, 02 minutes and 40 seconds
            \end{enumerate}
        \end{multicols}

    \item The solution to the ordinary differential equation
        $\frac{d^{2}y}{dx^{2}} + \frac{dy}{dx} - 6y = 0$ is
        \begin{multicols}{2}
            \begin{enumerate}
                \item $y=c_1e^{3x} + c_2e^{-2x}$
                \item $y=c_1e^{3x} + c_2e^{2x}$
                \item $y=c_1e^{-3x} + c_2e^{2x}$
                \item $y=c_1e^{-3x} + c_2e^{-2x}$
            \end{enumerate}
        \end{multicols}
\end{enumerate}
\end{document}
