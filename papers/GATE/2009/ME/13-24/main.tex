%iffalse
\documentclass[journal]{IEEEtran}
\usepackage[a5paper, margin=10mm]{geometry}
%\usepackage{lmodern} % Ensure lmodern is loaded for pdflatex
\usepackage{tfrupee} % Include tfrupee package


\setlength{\headheight}{1cm} % Set the height of the header box
\setlength{\headsep}{0mm}     % Set the distance between the header box and the top of the text


%\usepackage[a5paper, top=10mm, bottom=10mm, left=10mm, right=10mm]{geometry}

%
\usepackage{gvv-book}
\usepackage{gvv}
\setlength{\intextsep}{10pt} % Space between text and floats

\makeindex

\begin{document}
\bibliographystyle{IEEEtran}
\onecolumn
%endif

\title{2009 ME 13-24}
\author{AI24BTECH11031 - Shivram S}
\maketitle
\bigskip

\renewcommand{\thefigure}{\theenumi}
\renewcommand{\thetable}{\theenumi}

\begin{enumerate}
    \item The rotor shaft of a large electric motor supported between short bearings at both the ends
    shows a deflection of $1.8 mm$ in the middle of the rotor. Assuming the rotor to be perfectly balanced
    and supported at knife edges at both the ends, the likely critical speed (in rpm) of the shaft is

    \begin{multicols}{4}
    \begin{enumerate}
        \item 350
        \item 705
        \item 2810
        \item 4430
    \end{enumerate}
    \end{multicols}

    \item A solid circular shaft of diameter $d$ is subjected to a combined bending moment $M$ and torque
    $T$. The material property to be used for designing the shaft using the relation $\frac{16}{\pi d^3}\sqrt{M^2 + T^2}$
    is

    \begin{multicols}{2}
    \begin{enumerate}
        \item ultimate tensile stringth $\brak{S_u}$
        \item tensile yield strength $\brak{S_y}$
        \item torsional yield strength $\brak{S_{sy}}$
        \item endurance strength $\brak{S_e}$
    \end{enumerate}
    \end{multicols}

    \item The effective number of lattice points in the unit cell of simple cubic, body centered cubic,
    and face centered cubic space lattices, respectively, are
    \begin{multicols}{4}
    \begin{enumerate}
        \item 1, 2, 2
        \item 1, 2, 4
        \item 2, 3, 4
        \item 2, 4, 4
    \end{enumerate}
    \end{multicols}

    \item Friction at the tool-chip interface can be reduced by
    \begin{multicols}{2}
    \begin{enumerate}
        \item decreasing the rake angle
        \item increasing the depth of cut
        \item decreasing the cutting speed
        \item increasing the cutting speed
    \end{enumerate}
    \end{multicols}

    \item Two streams of liquid metal which are not hot enough to fuse properly
    result into a casting defect known as
    \begin{multicols}{4}
    \begin{enumerate}
        \item cold shut
        \item swell
        \item sand wash
        \item scab
    \end{enumerate}
    \end{multicols}

    \item The expected time $\brak{t_e}$ of a PERT activity in terms of optimistic time
    $\brak{t_o}$, pessimistic time $\brak{t_p}$ and most likely time $\brak{t_l}$
    is given by
    \begin{multicols}{2}
    \begin{enumerate}
        \item $\frac{t_o + 4t_l + t_p}{6}$
        \item $\frac{t_o + 4t_p + t_l}{6}$
        \item $\frac{t_o + 4t_l + t_p}{3}$
        \item $\frac{t_o + 4t_l + t_l}{3}$
    \end{enumerate}
    \end{multicols}

    \item Which of the following is the correct data structure for solid models?
    \begin{enumerate}
        \item solid part $\rightarrow$ faces $\rightarrow$ edges $\rightarrow$ vertices
        \item solid part $\rightarrow$ edges $\rightarrow$ faces $\rightarrow$ vertices
        \item vertices $\rightarrow$ edges $\rightarrow$ faces $\rightarrow$ solid parts
        \item vertices $\rightarrow$ faces $\rightarrow$ edges $\rightarrow$ solid parts
    \end{enumerate}

    \item Which of the following forecasting methods takes a fraction of forecast
    error into account for the next period forecast?
    \begin{multicols}{2}
    \begin{enumerate}
        \item simple average method
        \item moving average method
        \item weighted moving average method
        \item exponential smoothing method
    \end{enumerate}
    \end{multicols}

    \item An analytic function of a complex variable $z = x + iy$ is expressed as
    $f\brak{z} = u\brak{x, y} + iv\brak{x, y}$ where $i = \sqrt{-1}$. If $u = xy$, the expression
    for $v$ should be
    \begin{multicols}{4}
    \begin{enumerate}
        \item $\frac{\brak{x + y}^2}{2} + k$
        \item $\frac{x^2 - y^2}{2} + k$
        \item $\frac{y^2 - x^2}{2} + k$
        \item $\frac{\brak{x - y}^2}{2} + k$
    \end{enumerate}
    \end{multicols}

    \item The solution of $x\frac{dy}{dx} + y = x^4$ with the condition
    $y\brak{1} = \frac{6}{5}$ is
    \begin{multicols}{4}
    \begin{enumerate}
        \item $y = \frac{x^4}{5} + 1$
        \item $y = \frac{4x^4}{5} + \frac{4}{5x}$
        \item $y = \frac{x^4}{5} + 1$
        \item $y = \frac{x^5}{5} + 1$
    \end{enumerate}
    \end{multicols}

    \item A path $AB$ in the form of one quarter of a circle of unit radius is shown
    in the figure. Integration of $\brak{x + y}^2$ on path AB traversed in a
    counter-clockwise sense is

    \begin{center}
\begin{tikzpicture}
    \draw (2,0) arc[start angle=0,end angle=90,radius=2];
    \draw (0,0) -- (3,0) node[right] {$X$};
    \draw (0,0) -- (0,3) node[above] {$Y$}; 
    \node at (2,-0.3) {$A$};
    \node at (-0.3,2) {$B$};
\end{tikzpicture}
\end{center}

    \begin{multicols}{4}
    \begin{enumerate}
        \item $\frac{\pi}{2} - 1$
        \item $\frac{\pi}{2} + 1$
        \item $\frac{\pi}{2}$
        \item 1
    \end{enumerate}
    \end{multicols}
    
    \item The distance between the origin and the point nearest to it on the surface
    $z^2 = 1 + xy$ is
    \begin{multicols}{4}
    \begin{enumerate}
        \item 1
        \item $\frac{\sqrt{3}}{2}$
        \item $\sqrt{3}$
        \item 2
    \end{enumerate}
    \end{multicols}
\end{enumerate}

\end{document}
