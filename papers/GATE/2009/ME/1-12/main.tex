%iffalse
\documentclass[journal]{IEEEtran}
\usepackage[a5paper, margin=10mm]{geometry}
%\usepackage{lmodern} % Ensure lmodern is loaded for pdflatex
\usepackage{tfrupee} % Include tfrupee package


\setlength{\headheight}{1cm} % Set the height of the header box
\setlength{\headsep}{0mm}     % Set the distance between the header box and the top of the text


%\usepackage[a5paper, top=10mm, bottom=10mm, left=10mm, right=10mm]{geometry}

%
\usepackage{gvv-book}
\usepackage{gvv}
\setlength{\intextsep}{10pt} % Space between text and floats

\makeindex

\begin{document}
\bibliographystyle{IEEEtran}
\onecolumn
%endif

\title{2009 ME 1-12}
\author{AI24BTECH11031 - Shivram S}
\maketitle
\bigskip

\renewcommand{\thefigure}{\theenumi}
\renewcommand{\thetable}{\theenumi}

\begin{enumerate}
    \item For a matrix $\vec{M} = \myvec{\frac{3}{5} & \frac{4}{5} \\ x & \frac{3}{5}}$,
    the transpose of the matrix is equal to the inverse of the matrix, $\vec{M}^\top = \vec{M}^{-1}$.
    The value of $x$ is given by
    \begin{multicols}{4}
    \begin{enumerate}
        \item $-\frac{4}{5}$
        \item $-\frac{3}{5}$
        \item $\frac{3}{5}$
        \item $\frac{4}{5}$
    \end{enumerate}
    \end{multicols}

    \item The divergence of the vector field $3xz\hat{i} + 2xy\hat{j} - yz^2\hat{k}$
    at a point $\brak{1, 1, 1}$ is equal to
    \begin{multicols}{4}
    \begin{enumerate}
        \item 7
        \item 4
        \item 3
        \item 0
    \end{enumerate}
    \end{multicols}

    \item The inverse Laplace transform of $\frac{1}{\brak{s^2 + s}}$ is
    \begin{multicols}{4}
    \begin{enumerate}
        \item $1 + e^t$
        \item $1 - e^t$
        \item $1 - e^{-t}$
        \item $1 + e^{-t}$
    \end{enumerate}
    \end{multicols}

    \item If three coins are tossed simultaneously, the probability of getting at
    least one head is
    \begin{multicols}{4}
    \begin{enumerate}
        \item $\frac{1}{8}$
        \item $\frac{3}{8}$
        \item $\frac{1}{2}$
        \item $\frac{7}{8}$
    \end{enumerate}
    \end{multicols}

    \item If a closed system is undergoing an irreversible process, the entropy
    of the system
    \begin{enumerate}
        \item must increase
        \item always remains constant
        \item must decrease
        \item can increase, decrease or remain constant
    \end{enumerate}

    \item A coolant fluid at $30 \degree C$ flows over a heated flat plate
    maintained at a constant temperature of $100 \degree C$. The boundary
    layer temperature distribution at a given location on the plate may be
    approximated as $T = 30 + 70\exp\brak{-y}$ where $y$ (in m) is the
    distance normal to the plate and $T$ is in $\degree C$. If thermal
    conductivity of the fluid is 1.0 $\frac{W}{mK}$, the local convective heat
    transfer coefficient (in $\frac{W}{m^2K}$) at that location will be
    \begin{multicols}{4}
    \begin{enumerate}
        \item 0.2
        \item 1
        \item 5
        \item 10
    \end{enumerate}
    \end{multicols}

    \item A frictionless piston-cylinder device contains a gas initially at
    0.8 MPa and 0.015 $m^3$. It expands quasi-statically at constant
    temperature to a final volume of 0.030 $m^3$. The work output (in kJ)
    during this process will be
    \begin{multicols}{4}
    \begin{enumerate}
        \item 8.32
        \item 12.00
        \item 554.67
        \item 8320.00
    \end{enumerate}
    \end{multicols}

    \item In an ideal vapour compression refrigeration cycle, the specific
    enthalpy of refrigerant (in $\frac{kJ}{kg}$) at the following states is given as: \\
    Inlet of condenser: 283 \\
    Exit of condenser: 116 \\
    Exit of evaporator: 232 \\
    The COP of this cycle is
    \begin{multicols}{4}
    \begin{enumerate}
        \item 2.27
        \item 2.75
        \item 3.27
        \item 3.75
    \end{enumerate}
    \end{multicols}

    \item A compressor undergoes a reversible, steady flow process. The gas
    at inlet and outlet of the compressor is designated as state 1 and state
    2 respectively. Potential and kinetic energy changes are to be ignored.
    The following notations are used:
    
    $v$ = specific volume and $P$ = pressure of the gas.
    
    The specific work required to be supplied to the compressor for this gas
    compression process is 
    \begin{multicols}{4}
    \begin{enumerate}
    \item $\int_{1}^{2} Pdv$
    \item $\int_{1}^{2} vdP$
    \item $v_1\brak{P_2 - P_1}$
    \item $-P_2\brak{v_1 - v_2}$
    \end{enumerate}
    \end{multicols}

    \item A block weighing 981 N is resting on a horizontal surface. The coefficient
    of friction between the block and the horizontal surface is $\mu = 0.2$. A vertical
    cable attached to the block provides partial support as shown. A man can pull
    horizontally with a force of 100 N. What will be the tension, $T$ (in N), in the
    cable if the man is just able to move the block to the right?
    
    \begin{center}
\begin{tikzpicture}
    \draw[thick] (0,0) rectangle (2,2); 
    \draw[thick] (-1, 0) node[above] {$\mu = 0.2$} -- (3, 0); 
    
    \foreach \i in {-2,...,6} {
        \draw (\i/2, 0) -- (\i/2-0.2, -0.2);
    }

    \draw[->, thick] (2,1) -- (4,1) node[right] {100 N};
    \draw[->, thick] (1,2) -- (1,3) node[right] {T};
    \draw[dashed] (2,1) -- (1,1) node[below] {G} -- (1,2);
\end{tikzpicture}
\end{center}

    \begin{multicols}{4}
    \begin{enumerate}
        \item 176.2
        \item 196.0
        \item 481.0
        \item 981.0
    \end{enumerate}
    \end{multicols}

    \item If the principal stresses in a plane stress problem are $\sigma_1 = 100$ MPa,
    $\sigma_2 = 40$ MPa, the magnitude of the maximum shear stress (in MPa) will be
    \begin{multicols}{4}
    \begin{enumerate}
        \item 60
        \item 50
        \item 30
        \item 20
    \end{enumerate}
    \end{multicols}

    \item A simple quick return mechanism is shown in the figure. The forward to return
    ratio of the quick return mechanism is $2:1$. If the radius of the crank $O_1P$ is
    125 mm, then the distance $d$ (in mm) between the crank center to lever pivot center
    point should be

    \begin{center}
\begin{circuitikz}
    \draw (0, 2) to[R=3.77 $\Omega$] (2, 2) to[L=10 mH] (4, 2) to[switch,l_={$t=0$}] (6, 2)
    -- (6, 0) -- (0, 0) to[sV,l_={$v(t) = 150\sin\brak{377t + \theta}$}] (0, 2);
\end{circuitikz}
\end{center}

    \begin{multicols}{4}
    \begin{enumerate}
        \item 144.3
        \item 216.5
        \item 240.0
        \item 250.0
    \end{enumerate}
    \end{multicols}
\end{enumerate}
\end{document}
