%iffalse
\documentclass[journal]{IEEEtran}
\usepackage[a5paper, margin=10mm]{geometry}
%\usepackage{lmodern} % Ensure lmodern is loaded for pdflatex
\usepackage{tfrupee} % Include tfrupee package


\setlength{\headheight}{1cm} % Set the height of the header box
\setlength{\headsep}{0mm}     % Set the distance between the header box and the top of the text


%\usepackage[a5paper, top=10mm, bottom=10mm, left=10mm, right=10mm]{geometry}

%
\usepackage{gvv-book}
\usepackage{gvv}
\setlength{\intextsep}{10pt} % Space between text and floats

\makeindex

\begin{document}
\bibliographystyle{IEEEtran}
\onecolumn
%endif

\title{2021-February Session-02-26-2021-shift-1-16-30}
\author{AI24BTECH11031 - Shivram S}
\maketitle
\bigskip

\renewcommand{\thefigure}{\theenumi}
\renewcommand{\thetable}{\theenumi}

\begin{enumerate}
    \item The value of $\lim\limits_{h \to 0} 2\cbrak{\frac{\sqrt{3}\sin\brak{\frac{\pi}{6} - h} - \cos\brak{\frac{\pi}{6} + h}}{\sqrt{3}h\brak{\sqrt{3}\cos h - \sin h}}}$ is
    
    \begin{multicols}{4}
    \begin{enumerate}
        \item $\frac{3}{4}$
        \item $\frac{2}{\sqrt{3}}$
        \item $\frac{4}{3}$
        \item $\frac{2}{3}$
    \end{enumerate}
    \end{multicols}

    \item A fair coin is tossed a fixed number of times. If the probability of
    getting 7 heads is equal to the probability of getting 9 heads, then the
    probability of getting 2 heads is:
    
    \begin{multicols}{4}
    \begin{enumerate}
        \item $\frac{15}{2^{12}}$
        \item $\frac{15}{2^{13}}$
        \item $\frac{15}{2^{14}}$
        \item $\frac{15}{2^{8}}$
    \end{enumerate}
    \end{multicols}

    \item If $(1, 5, 35)$, $(7, 5, 5)$, $(1, \lambda, 7)$ and $(2\lambda, 1, 2)$ are
    coplanar, then the sum of all possible values of $\lambda$ is:

    \begin{multicols}{4}
    \begin{enumerate}
        \item $-\frac{44}{5}$
        \item $\frac{39}{5}$
        \item $-\frac{39}{5}$
        \item $\frac{44}{5}$
    \end{enumerate}
    \end{multicols}

    \item Let $R = \cbrak{(P,Q) \mid \text{P and Q are at the same distance from the origin}}$
    be a relation, then the equivalence class of $(1,-1)$ is the set:

    \begin{multicols}{2}
    \begin{enumerate}
        \item $S = \cbrak{(x, y) \mid x^2 + y^2 = 1}$
        \item $S = \cbrak{(x, y) \mid x^2 + y^2 = 4}$
        \item $S = \cbrak{(x, y) \mid x^2 + y^2 = \sqrt{2}}$
        \item $S = \cbrak{(x, y) \mid x^2 + y^2 = 2}$
    \end{enumerate}
    \end{multicols}

    \item In the circle given below, let $OA$ = 1 unit,
    $OB$ = 13 unit and $PQ$ perpendicular to $OB$. Then, the area
    of the triangle $PQB$ (in square units) is:

    \begin{center}
    \begin{tikzpicture}
        \draw [->] (0, -2) -- (0, 2) node[left] {Y};
        \draw [->] (-2, 0) -- (4, 0) node[right] {X};
        \draw (1, 0) circle (1);
        \draw (2, 0) node[below right] {B} --
                (0.5, -0.86) node[below] {Q} --
                (0.5, 0.86) node[above] {P} -- (2, 0);
        \draw (0, 0) node[below left] {O} -- (0.5, 0) node[below right] {A};
    \end{tikzpicture}
    \end{center} 


    \begin{multicols}{4}
    \begin{enumerate}
        \item $26 \sqrt{3}$
        \item $24 \sqrt{2}$
        \item $24 \sqrt{3}$
        \item $26 \sqrt{2}$
    \end{enumerate}
    \end{multicols}
    
    \item The area bounded by the lines $y = \abs{x - 1} - 2$ is \rule{1cm}{0.15mm}.
    
    \item The number of integral values of $k$ for which the equation
    $3\sin x + 4\cos x = k + 1$ has a solution, $k \in \mathbb{R}$ is \rule{1cm}{0.15mm}.

    \item Let $m, n \in \mathbb{N}$ and $\gcd(2, n) = 1$. If
    $30\binom{30}{0} + 29\binom{30}{1} + \dots + 2\binom{30}{28} + 1\binom{30}{29} = n \cdot 2^m$,
    then $n + m = \rule{1cm}{0.15mm}$.

    \item If $y = y(x)$ is the solution of the equation $e^{\sin y}\cos y \frac{dy}{dx} + e^{\sin y}\cos x = \cos x$,
    $y(0) = 0$; then $1 + y\brak{\frac{\pi}{6}} + \frac{\sqrt{3}}{2} y\brak{\frac{\pi}{3}} + \frac{1}{\sqrt{2}} y\brak{\frac{\pi}{4}}$
    is equal to \rule{1cm}{0.15mm}.

    \item The number of solutions of the equation $\log_4(x - 1) = \log_2(x - 3)$ is \rule{1cm}{0.15mm}.
    
    \item If $\sqrt{3}(\cos^2 x) = (\sqrt{3} - 1) \cos x + 1$, the number of solutions
    of the given equation when $x \in \sbrak{0, \frac{\pi}{2}}$ is \rule{1cm}{0.15mm}.

    \item Let $(\lambda, 2, 1)$ be a point on the plane which passes through the point
    $(4, -2, 2)$. If the plane is perpendicular to the line joining the points
    $(-2, -21, 29)$ and $(-1, -16, 23)$, then $\brak{\frac{\lambda}{11}}^2 - \frac{4\lambda}{11} - 4$
    is equal to \rule{1cm}{0.15mm}.

    \item The difference between degree and order of a differential equation that
    represents the family of curves given by $y^2 = a \brak{x + \frac{\sqrt{a}}{2}}$,
    $a > 0$ is \rule{1cm}{0.15mm}.

    \item The sum of $162^{th}$ power of the roots of the equation
    $x^3 - 2x^2 + 2x - 1 = 0$ is \rule{1cm}{0.15mm}.

    \item The value of the integral $\int_0^\pi \abs{\sin 2x} dx$ is \rule{1cm}{0.15mm}.

\end{enumerate}

\end{document}
