%iffalse
\documentclass[journal]{IEEEtran}
\usepackage[a5paper, margin=10mm]{geometry}
%\usepackage{lmodern} % Ensure lmodern is loaded for pdflatex
\usepackage{tfrupee} % Include tfrupee package


\setlength{\headheight}{1cm} % Set the height of the header box
\setlength{\headsep}{0mm}     % Set the distance between the header box and the top of the text


%\usepackage[a5paper, top=10mm, bottom=10mm, left=10mm, right=10mm]{geometry}

%
\usepackage{gvv-book}
\usepackage{gvv}
\setlength{\intextsep}{10pt} % Space between text and floats

\makeindex

\begin{document}
\bibliographystyle{IEEEtran}
\onecolumn
%endif

\title{2021-February Session-02-26-2021-shift-1-21-30}
\author{AI24BTECH11031 - Shivram S}
\maketitle
\bigskip

\renewcommand{\thefigure}{\theenumi}
\renewcommand{\thetable}{\theenumi}

\begin{enumerate}
    \item The area bounded by the lines $y = \abs{x - 1} - 2$ is \rule{1cm}{0.15mm}.
    \hfill{[Feb 2021]}
    
    \item The number of integral values of $k$ for which the equation
    $3\sin x + 4\cos x = k + 1$ has a solution, $k \in \mathbb{R}$ is \rule{1cm}{0.15mm}.
    \hfill{[Feb 2021]}

    \item Let $m, n \in \mathbb{N}$ and $\gcd(2, n) = 1$. If
    $30\binom{30}{0} + 29\binom{30}{1} + \dots + 2\binom{30}{28} + 1\binom{30}{29} = n \cdot 2^m$,
    then $n + m = \rule{1cm}{0.15mm}$.
    \hfill{[Feb 2021]}

    \item If $y = y(x)$ is the solution of the equation $e^{\sin y}\cos y \frac{dy}{dx} + e^{\sin y}\cos x = \cos x$,
    $y(0) = 0$; then $1 + y\brak{\frac{\pi}{6}} + \frac{\sqrt{3}}{2} y\brak{\frac{\pi}{3}} + \frac{1}{\sqrt{2}} y\brak{\frac{\pi}{4}}$
    is equal to \rule{1cm}{0.15mm}.
    \hfill{[Feb 2021]}

    \item The number of solutions of the equation $\log_4(x - 1) = \log_2(x - 3)$ is \rule{1cm}{0.15mm}.
    \hfill{[Feb 2021]}
    
    \item If $\sqrt{3}(\cos^2 x) = (\sqrt{3} - 1) \cos x + 1$, the number of solutions
    of the given equation when $x \in \sbrak{0, \frac{\pi}{2}}$ is \rule{1cm}{0.15mm}.
    \hfill{[Feb 2021]}

    \item Let $(\lambda, 2, 1)$ be a point on the plane which passes through the point
    $(4, -2, 2)$. If the plane is perpendicular to the line joining the points
    $(-2, -21, 29)$ and $(-1, -16, 23)$, then $\brak{\frac{\lambda}{11}}^2 - \frac{4\lambda}{11} - 4$
    is equal to \rule{1cm}{0.15mm}.
    \hfill{[Feb 2021]}

    \item The difference between degree and order of a differential equation that
    represents the family of curves given by $y^2 = a \brak{x + \frac{\sqrt{a}}{2}}$,
    $a > 0$ is \rule{1cm}{0.15mm}.
    \hfill{[Feb 2021]}

    \item The sum of $162^{th}$ power of the roots of the equation
    $x^3 - 2x^2 + 2x - 1 = 0$ is \\
    \rule{1cm}{0.15mm}. \hfill{[Feb 2021]}

    \item The value of the integral $\int_0^\pi \abs{\sin 2x} dx$ is \rule{1cm}{0.15mm}.
    \hfill{[Feb 2021]}

\end{enumerate}

\end{document}
