%iffalse
\let\negmedspace\undefined
\let\negthickspace\undefined
\documentclass[journal,12pt,twocolumn]{IEEEtran}
\usepackage{cite}
\usepackage{amsmath,amssymb,amsfonts,amsthm}
\usepackage{algorithmic}
\usepackage{graphicx}
\usepackage{textcomp}
\usepackage{xcolor}
\usepackage{txfonts}
\usepackage{listings}
\usepackage{enumitem}
\usepackage{mathtools}
\usepackage{gensymb}
\usepackage{comment}
\usepackage[breaklinks=true]{hyperref}
\usepackage{tkz-euclide} 
\usepackage{listings}
\usepackage{gvv}                                        
\def\inputGnumericTable{}                                 
\usepackage[latin1]{inputenc}                                
\usepackage{color}                                            
\usepackage{array}                                            
\usepackage{longtable}                                       
\usepackage{calc}                                             
\usepackage{multirow}                                         
\usepackage{hhline}                                           
\usepackage{ifthen}                                           
\usepackage{lscape}

\newtheorem{theorem}{Theorem}[section]
\newtheorem{problem}{Problem}
\newtheorem{proposition}{Proposition}[section]
\newtheorem{lemma}{Lemma}[section]
\newtheorem{corollary}[theorem]{Corollary}
\newtheorem{example}{Example}[section]
\newtheorem{definition}[problem]{Definition}
\newcommand{\BEQA}{\begin{eqnarray}}
\newcommand{\EEQA}{\end{eqnarray}}
\newcommand{\define}{\stackrel{\triangle}{=}}
\theoremstyle{remark}
\newtheorem{rem}{Remark}

\newcommand{\RE}{\mathrm{Re}}
\newcommand{\IM}{\mathrm{Im}}

\begin{document}

\bibliographystyle{IEEEtran}
\vspace{3cm}
%fi

\title{Assignment 1}
\author{AI24BTECH11031 - Shivram S$^{*}$% <-this % stops a space
}
\maketitle
\newpage
\bigskip

\renewcommand{\thefigure}{\theenumi}
\renewcommand{\thetable}{\theenumi}
\section{Match The Following}

\textbf{Question 1:} $z \ne 0$ is a complex number
\hspace*{\fill} {\color{magenta}\brak{1992 - 2\ marks}}

\begin{tabular}{ c c c c }
		& \textbf{Column I}			&		& \textbf{Column II}		\\
	(A) & $\RE\ z = 0$				& (p)	& $\RE\ z^2  = 0$			\\
	(B) & $\mathrm{Arg}\ z = \frac\pi4$		& (q)	& $\IM\ z^2  = 0$	\\
		&							& (r)	& $\RE\ z^2  = \IM\ z^2$
\end{tabular}

\textbf{Question 2:} Match the statements in \textbf{Column I} with those in \textbf{Column II}
\hspace*{\fill} {\color{magenta}\brak{2010}} \\
\textbf{[Note:} here $z$ takes values in the complex plane and $\IM\ z$ and $\RE\ z$
denote, respectively, the imaginary part and the real part of $z$. \textbf{]}

\textbf{Column I}
\begin{enumerate}[label=(\Alph*)]
	\item The set of points $z$ satisfying $|z - i|z||$ = $|z + i|z||$ is contained in or equal to
	\item The set of points $z$ satisfying $|z + 4| + |z - 4| = 10$ is contained in or equal to
	\item If $|w| = 2$, then the set of points $z = w - \frac 1 w$ is contained in or equal to
	\item If $|w| = 1$, then the set of points $z = w - \frac 1 w$ is contained in or equal to
\end{enumerate}

\begin{flushright}
\textbf{Column II}

\begin{enumerate}[label=(\alph*)]
\setcounter{enumi}{15}
\item an ellipse with eccentricity $\frac 4 5$
\item the set of points $z$ satisfying $\IM\ z = 0$
\item the set of points $z$ satisfying $|\IM\ z| \le 1$
\item the set of points $z$ satisfying $|\RE\ z| < 2$
\item the set of points $z$ satisfying $|z| \le 3$
\end{enumerate}
\end{flushright}

\textbf{Question 3:} Let $z_k = \cos\brak{\frac {2k\pi} {10}} + i\sin\brak{\frac {2k\pi} {10}}$; $k = 1,2,\dots,9$.
\hspace*{\fill} {\color{magenta}\brak{JEE\ Adv.\ 2014}} \\

\textbf{List I} \\
\textbf P. For each $z_k$ there exists a $z_j$ such that $z_k \cdot z_j = 1$ \\
\textbf Q. There exists a $k \in \{1,2,\dots, 9\}$ such that $z_1 \cdot z = z_k$ has no 
solution in the set of complex numbers \\
\textbf R. $\frac {|1 - z_1||1 - z_2|\dots|1-z_9|} {10}$ equals \\
\textbf S. $1 - \sum\limits_{k=1}^9 \cos \brak{\frac {2k\pi} {10}}$ equals \\

\begin{flushright}
\textbf{List II} \\
\textbf 1. True \\
\textbf 2. False \\
\textbf 3. 1 \\
\textbf 4. 2 \\
\end{flushright}

\begin{tabular}{c c c c c c c c c c}
	& \textbf P & \textbf Q & \textbf R & \textbf S & & \textbf P & \textbf Q & \textbf R & \textbf S \\
	(a) & 1 & 2 & 4 & 3 & (b) & 2 & 1 & 3 & 4 \\
	(c( & 1 & 2 & 3 & 4 & (d) & 2 & 1 & 4 & 3 \\
\end{tabular}

\section{Comprehension Based Questions}

\subsection{Passage-2}

Let $S = S_1 \cap S_2 \cap S_3$ where
\begin{gather*}
S_1 = \{z \in \mathbb C: |z| < 4 \} \\
	S_2 = \left\{z \in \mathbb C: \IM\left[ \frac {z - 1 + \sqrt 3 i} {1 - \sqrt 3 i} \right] > 0 \right\} \\
	\text{and} \ S_3 = \{z \in \mathbb C: \RE\ z > 0\}
\end{gather*}

\textbf{Question 4:} Area of S = 
\hspace*{\fill} {\color{magenta}\brak{JEE\ Adv.\ 2013}} \\

\begin{tabular}{ c c c c }
	(a) $\frac {10\pi} 3$ & (b) $\frac {20\pi} 3$ & (c) $\frac {16\pi} 3$ & (d) $\frac {32\pi} 3$
\end{tabular}

\textbf{Question 5:} $\min\limits_{z \in S} |1 - 3i - z| = $
\hspace*{\fill} {\color{magenta}\brak{JEE\ Adv.\ 2013}} \\

\begin{tabular}{ c c }
	(a) $\frac {2 - \sqrt 3} 2$ & (b) $\frac {2 + \sqrt 3} 2$ \\
	(b) $\frac {3 -\sqrt 3} 2$ & (d) $\frac {3 + \sqrt 3} 2$
\end{tabular}

\section{Integer Value Correct Type}

\textbf{Question 1:} If $z$ is any complex number satisfying $|z - 3 - 2i| < 2$, then the
minimum value of $|2z - 6 + 5i|$ is
\hspace*{\fill} {\color{magenta}\brak{2011}} \\

\textbf{Question 2:} Let $\omega = e^{\frac{i\pi}3}$, and $a, b, c, x, y, z$ be non-zero complex numbers such that:
\hspace*{\fill} {\color{magenta}\brak{2011}} \\
\begin{gather*}
	a + b + c = x \\
	a + b\omega + c\omega^2 = y \\
	a + b\omega^2 + c\omega = z 
\end{gather*}
Then the value of $\frac {|x|^2 + |y|^2 + |z|^2} {|a|^2 + |b|^2 + |c|^2}$ is

\textbf{Question 3:} For any integer $k$, let $a_k = \cos(\frac{k\pi}7) + i\sin(\frac{k\pi}7)$, where
$i = \sqrt{-1}$. The value of the expression $\frac {\sum_{k=1}^{12} |a_{k+1} - a_k|} {\sum_{k=1}^{3} |a_{4k-1} - a_{4k-2}}$ is
\hspace*{\fill} {\color{magenta}\brak{JEE\ Adv.\ 2015}} \\


\textbf{Question 4:} Let $\omega \ne 1$ be a cube root of unity. Then the minimum of the set $\left\{|a + b\omega + c\omega^2|^2: a,b,c \text{ distinct non-zero integers} \right\}$ equals \rule{1cm}{0.15mm}.
\hspace*{\fill} {\color{magenta}\brak{JEE\ Adv.\ 2019}} \\


\end{document}
