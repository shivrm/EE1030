%iffalse
\let\negmedspace\undefined
\let\negthickspace\undefined
\documentclass[journal,12pt,twocolumn]{IEEEtran}
\usepackage{cite}
\usepackage{amsmath,amssymb,amsfonts,amsthm}
\usepackage{algorithmic}
\usepackage{graphicx}
\usepackage{textcomp}
\usepackage{xcolor}
\usepackage{txfonts}
\usepackage{listings}
\usepackage{enumitem}
\usepackage{mathtools}
\usepackage{gensymb}
\usepackage{comment}
\usepackage[breaklinks=true]{hyperref}
\usepackage{tkz-euclide} 
\usepackage{listings}
\usepackage{gvv}                                        
\def\inputGnumericTable{}                                 
\usepackage[latin1]{inputenc}                                
\usepackage{color}                                            
\usepackage{array}                                            
\usepackage{longtable}                                       
\usepackage{calc}                                             
\usepackage{multirow}                                         
\usepackage{hhline}                                           
\usepackage{ifthen}                                           
\usepackage{lscape}

\newtheorem{theorem}{Theorem}[section]
\newtheorem{problem}{Problem}
\newtheorem{proposition}{Proposition}[section]
\newtheorem{lemma}{Lemma}[section]
\newtheorem{corollary}[theorem]{Corollary}
\newtheorem{example}{Example}[section]
\newtheorem{definition}[problem]{Definition}
\newcommand{\BEQA}{\begin{eqnarray}}
\newcommand{\EEQA}{\end{eqnarray}}
\newcommand{\define}{\stackrel{\triangle}{=}}
\theoremstyle{remark}
\newtheorem{rem}{Remark}

\usepackage{multicol}
\begin{document}

\bibliographystyle{IEEEtran}
%fi

\title{Assignment 2}
\author{AI24BTECH11031 - Shivram S}
\maketitle
\newpage
\bigskip

\renewcommand{\thefigure}{\theenumi}
\renewcommand{\thetable}{\theenumi}

\section{Fill in The Blanks}
\begin{enumerate}[label=\arabic*.]
	\item Let $p\lambda^2 + q\lambda^3 + r\lambda^2 + s\lambda + t =
		\mydet{
			\lambda^2 + 3\lambda & \lambda - 1 & \lambda + 3 \\
			\lambda + 1 & -2\lambda & \lambda - 4 \\
			\lambda - 3 & \lambda+ 4 & 3\lambda
		}$ be an identity in $\lambda$ where $p, q, r, s$ and $t$ are constants.
		Then the value of $t$ is \rule{1cm}{0.15mm}.
		\hfill \brak{1981 - 2\ Marks}
	
	\item The solution set of the equation $\mydet{
			1 & 4 & 20 \\
			1 & -2 & 5 \\
			1 & 2x & 5x^2
		} = 0$ is \rule{1cm}{0.15mm}.
		\hfill \brak{1981 - 2\ Marks}

	\item A determinant is chosen at random from the set of all determinants of order
		2 with elements 0 or 1 only. The probability that the value of the determinant
		chosen is positive is \rule{1cm}{0.15mm}.
		\hfill \brak{1982 - 2\ Marks}

	\item Given that $x = -9$ is a root of $\mydet{
			x & 3 & 7 \\
			2 & x & 2 \\
			7 & 6 & x
		}$ the other two roots are \rule{1cm}{0.15mm} and \rule{1cm}{0.15mm}.
		\hfill \brak{1983 - 2\ Marks}

	\item The system of equations
		\begin{gather*}
			\lambda x + y + z = 0 \\
			-x + \lambda y + z = 0 \\
			-x - y + \lambda x = 0
		\end{gather*}
		Will have a non-zero solution if real values of $\lambda$ are given by
		\rule{1cm}{0.15mm}.
		\hfill \brak{1984 - 2\ Marks}

	\item The value of the determinant $\mydet{
			1 & a & a^2 - bc \\
			1 & b & b^2 - ca \\
			1 & c & c^2 - ab
		}$ is \rule{1cm}{0.15mm}.	
		\hfill \brak{1988 - 2\ Marks}

	\item For positive numbers $x, y$ and $z$, the numerical value of the determinant
		$\mydet{
			1 & \log_x{y} & \log_x{z} \\
			\log_y{x} & 1 & \log_y{z} \\
			\log_z{x} & \log_z{y} & 1 \\
		}$ is \rule{1cm}{0.15mm}.
		\hfill \brak{1993 - 2\ Marks}
\end{enumerate}

\section{True / False}

\begin{enumerate}[label=\arabic*.]
	\item The determinants $\mydet{
			1 & a & bc \\
			1 & b & ca \\
			1 & c & ab
		}$ and $\mydet{
			1 & a & a^2 \\
			1 & b & b^2 \\
			1 & c & c^2
		}$ are not identically equal.
		\hfill \brak{1983 - 1\ Mark}
	
	\item If $\mydet{
			x_1 & y_1 & 1 \\
			x_2 & y_2 & 1 \\
			x_3 & y_3 & 1
		} = \mydet {
			a_1 & b_1 & 1 \\
			a_2 & b_2 & 1 \\
			a_3 & b_3 & 1
		}$ then the two triangles with vertices	
		$\brak{x_1, y_1}, \brak{x_2, y_2}, \brak{x_3, y_3}$,
		and $\brak{a_1, b_1}, \brak{a_2, b_2}, \brak{a_3, b_3}$
		must be congruent.
		\hfill \brak{1985 - 1\ Mark}
\end{enumerate}

\section{MCQs with One Corrrect Answer}

\begin{enumerate}[label=\arabic*.]
	\item Consider the set $A$ of all determinants of order 3 with entries
		0 or 1 only. Let $B$ be the subset of $A$ consisting of all
		determinants with value 1. Let $C$ be the subset of $A$ consisting
		of all determinants with value -1. Then
		\hfill \brak{1981 - 2\ Marks}
		\begin{enumerate}[label=(\alph*)]
			\item $C$ is empty
			\item $B$ has as many elements as $C$
			\item $A = B \cup C$
			\item $B$ has twice as many elements as $C$
		\end{enumerate}

	\item If $\omega (\ne 1)$ is a cube root of unity, then
		$\mydet{
			1 & 1 + i + \omega^2 & \omega^2 \\
			1 - i & -1 & \omega^2 - 1 \\
			-i & -i + \omega - 1 & -1
		} = $
		\hfill \brak{1995S}
	
		\begin{multicols}{4}
			\begin{enumerate}[label=(\alph*)]
				\item 0
				\item 1
				\item $i$
				\item $\omega$
			\end{enumerate}
		\end{multicols}

	\item Let $a, b, c$ be the real numbers. Then following system of
		equations in $x, y$ and $z$ 
		\hfill \brak{1995S}
		$
		\frac{x^2}{a^2} + \frac{y^2}{b^2} - \frac{z^2}{c^2} = 1	,
		\frac{x^2}{a^2} - \frac{y^2}{b^2} + \frac{z^2}{c^2} = 1	,
		-\frac{x^2}{a^2} + \frac{y^2}{b^2} + \frac{z^2}{c^2} = 1
		$ has

		\begin{multicols}{2}
			\begin{enumerate}[label=(\alph*)]
				\item no solution
				\item unique solution
				\columnbreak
				\item infinitely many solutions
				\item finitely many solutions
			\end{enumerate}
		\end{multicols}

	\item If $A$ and $B$ are square matrices of equal degree, then which
		one is correct among the followings?
		\hfill \brak{1995S}

		\begin{multicols}{2}
			\begin{enumerate}[label=(\alph*)]
				\item $A + B = B + A$
				\item $A + B = A - B$
				\columnbreak
				\item $A - B = B - A$
				\item $AB = BA$
			\end{enumerate}
		\end{multicols}

	\item The parameter on which the value of the determinant $\mydet{
			1 & a & a^2 \\
			\cos\brak{p-d}x & \cos px & \cos\brak{p + d}x \\
			\sin\brak{p-d}x & \sin px & \sin\brak{p + d}x	
		}$ does not depend upon is
		\hfill \brak{1997 - 2\ Marks}

		\begin{multicols}{4}
			\begin{enumerate}[label=(\alph*)]
				\item $a$
				\item $p$
				\item $d$
				\item $x$
			\end{enumerate}
		\end{multicols}

	\item If $f(x) = \mydet{
			1 & x & x + 1 \\
			2x & x(x - 1) & (x + 1)x \\
			3x(x - 1) & x(x - 1)(x - 2) & (x + 1)x(x - 1)
		}$ then $f(100)$ is equal to
		\hfill \brak{1999 - 2\ Marks}

		\begin{multicols}{4}
			\begin{enumerate}[label=(\alph*)]
				\item 0
				\item 1
				\item 100
				\item -100
			\end{enumerate}
		\end{multicols}

\end{enumerate}

\end{document}
