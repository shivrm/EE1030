%iffalse
\let\negmedspace\undefined
\let\negthickspace\undefined
\documentclass[journal,12pt,twocolumn]{IEEEtran}
\usepackage{cite}
\usepackage{amsmath,amssymb,amsfonts,amsthm}
\usepackage{algorithmic}
\usepackage{graphicx}
\usepackage{textcomp}
\usepackage{xcolor}
\usepackage{txfonts}
\usepackage{listings}
\usepackage{enumitem}
\usepackage{mathtools}
\usepackage{gensymb}
\usepackage{comment}
\usepackage[breaklinks=true]{hyperref}
\usepackage{tkz-euclide} 
\usepackage{listings}
\usepackage{gvv}                                        
\def\inputGnumericTable{}                                 
\usepackage[latin1]{inputenc}                                
\usepackage{color}                                            
\usepackage{array}                                            
\usepackage{longtable}                                       
\usepackage{calc}                                             
\usepackage{multirow}                                         
\usepackage{hhline}                                           
\usepackage{ifthen}                                           
\usepackage{lscape}

\newtheorem{theorem}{Theorem}[section]
\newtheorem{problem}{Problem}
\newtheorem{proposition}{Proposition}[section]
\newtheorem{lemma}{Lemma}[section]
\newtheorem{corollary}[theorem]{Corollary}
\newtheorem{example}{Example}[section]
\newtheorem{definition}[problem]{Definition}
\newcommand{\BEQA}{\begin{eqnarray}}
\newcommand{\EEQA}{\end{eqnarray}}
\newcommand{\define}{\stackrel{\triangle}{=}}
\theoremstyle{remark}
\newtheorem{rem}{Remark}

\usepackage{multicol}
\begin{document}

\bibliographystyle{IEEEtran}
%fi

\title{Assignment 1}
\author{AI24BTECH11031 - Shivram S}
\maketitle
\newpage
\bigskip

\renewcommand{\thefigure}{\theenumi}
\renewcommand{\thetable}{\theenumi}
\section{Match The Following}

\begin{enumerate}
	\item $z \ne 0$ is a complex number
		\hfill \brak{1992 - 2\ marks}

		\begin{multicols}{2}
			\textbf{Column I}
			\begin{enumerate}[label=(\Alph*)]
				\item $\mathrm{Re}\ z = 0$
				\item $\mathrm{Arg}\ z = \frac{\pi}{4}$
			\end{enumerate}
			\columnbreak
			\textbf{Column II}
			\begin{enumerate}[label=(\alph*)]
					\setcounter{enumii}{15}
				\item $\mathrm{Re}\ z^2 = 0$
				\item $\mathrm{Im}\ z^2 = 0$
				\item $\mathrm{Re}\ z^2 = \mathrm{Im}\ z^2$
			\end{enumerate}
		\end{multicols}

	\item Match the statements in \textbf{Column I} with those in \textbf{Column II}
		\hfill \brak{2010} \\
		\textbf{[Note:} here $z$ is a set of points taking values in the
		complex plane and $\mathrm{Im}\ z$ and $\mathrm{Re}\ z$
		denote, respectively, the imaginary part and the real part of $z$. \textbf{]}

		\textbf{Column I}
		\begin{enumerate}[label=(\Alph*)]
			\item The set of points $z$ satisfying $\abs{z - i\abs{z}}$ = $\abs{z + i\abs{z}}$ is in or equals
			\item The set of points $z$ satisfying $\abs{z + 4} + \abs{z - 4} = 10$ is in or equals
			\item If $\abs{w} = 2$, then the set of points $z = w - \frac{1}{w}$ is in or equals
			\item If $\abs{w} = 1$, then the set of points $z = w - \frac{1}{w}$ is in or equals
		\end{enumerate}

		\textbf{Column II}
		\begin{enumerate}[label=(\alph*)]
				\setcounter{enumii}{15}
			\item an ellipse with eccentricity $\frac{4}{5}$
			\item the set of points $z$ satisfying $\mathrm{Im}\ z = 0$
			\item the set of points $z$ satisfying $\abs{\mathrm{Im}\ z} \le 1$
			\item the set of points $z$ satisfying $\abs{\mathrm{Re}\ z} < 2$
			\item the set of points $z$ satisfying $\abs{z} \le 3$
		\end{enumerate}

	\item Let $z_k = \cos\brak{\frac{2k\pi}{10}} + i\sin\brak{\frac{2k\pi}{10}}$; $k = 1,2,\dots,9$.
		\hfill \brak{JEE\ Adv.\ 2014}

		\textbf{List I}
		\begin{enumerate}[label=\Alph*.]
				\setcounter{enumii}{15}
			\item For each $z_k$ there exists a $z_j$ such that $z_k \cdot z_j = 1$
			\item There exists a $k \in \cbrak{1,2,\dots, 9}$ such that $z_1 \cdot z = z_k$ has no 
				solution in the set of complex numbers
			\item $\frac {\abs{1 - z_1}\abs{1 - z_2}\dots\abs{1-z_9}} {10}$ equals
			\item $1 - \sum\limits_{k=1}^9 \cos \brak{\frac {2k\pi} {10}}$ equals
		\end{enumerate}

		\textbf{List II}
		\begin{enumerate}[label=\arabic*.]
			\item True
			\item False
			\item 1
			\item 2
		\end{enumerate}

		\begin{tabular}{c c c c c c c c c c}
			& \textbf{P} & \textbf{Q} & \textbf{R} & \textbf{S} & & \textbf{P} & \textbf{Q} & \textbf{R} & \textbf{S} \\
			\brak{a} & 1 & 2 & 4 & 3 & \brak{b} & 2 & 1 & 3 & 4 \\
			\brak{c} & 1 & 2 & 3 & 4 & \brak{d} & 2 & 1 & 4 & 3 \\
		\end{tabular}
\end{enumerate}

\section{Comprehension Based Questions}
\subsection{Passage-2}

Let $S = S_1 \cap S_2 \cap S_3$ where
\begin{gather*}
	S_1 = \cbrak{z \in \mathbb C: \abs{z} < 4 } \\
	S_2 = \cbrak{z \in \mathbb C: \mathrm{Im} \sbrak{ \frac {z - 1 + \sqrt 3 i} {1 - \sqrt 3 i} } > 0 } \\
	\text{and} \ S_3 = \{z \in \mathbb C: \mathrm{Re}\ z > 0\}
\end{gather*}

\begin{enumerate}
		\setcounter{enumi}{3}
	\item Area of S = 
		\hfill \brak{JEE\ Adv.\ 2013}

		\begin{multicols}{4}
			\begin{enumerate}
				\item $\frac {10\pi} {3}$ 
				\item $\frac {20\pi} {3}$
				\item $\frac {16\pi} {3}$
				\item $\frac {32\pi} {3}$
			\end{enumerate}
		\end{multicols}

	\item $\min\limits_{z \in S} \abs{1 - 3i - z} = $
		\hfill \brak{JEE\ Adv.\ 2013}

		\begin{multicols}{2}
			\begin{enumerate}
				\item $\frac {2 - \sqrt{3}} {2}$
				\item $\frac {2 + \sqrt{3}} {2}$
				\item $\frac {3 -\sqrt{3}} {2}$
				\item $\frac {3 + \sqrt{3}} {2}$
			\end{enumerate}
		\end{multicols}
\end{enumerate}

\section{Integer Value Correct Type}

\begin{enumerate}
	\item If $z$ is any complex number satisfying $\abs{z - 3 - 2i} < 2$, then the
		minimum value of $\abs{2z - 6 + 5i}$ is
		\hfill \brak{2011}

	\item Let $\omega = e^{\frac{i\pi}3}$, and $a, b, c, x, y, z$ be non-zero complex numbers such that:
		\hfill \brak{2011}
		\begin{gather*}
			a + b + c = x \\
			a + b\omega + c\omega^2 = y \\
			a + b\omega^2 + c\omega = z 
		\end{gather*}

		Then the value of $\frac {\abs{x}^2 + \abs{y}^2 + \abs{z}^2} {\abs{a}^2 + \abs{b}^2 + \abs{c}^2}$ is

	\item For any integer $k$, let $a_k = \cos\brak{\frac{k\pi}{7}} + i\sin\brak{\frac{k\pi}{7}}$, where	
		$i = \sqrt{-1}$. The value of the expression $\frac {\sum_{k=1}^{12} \abs{a_{k+1} - a_k}} {\sum_{k=1}^{3} \abs{a_{4k-1} - a_{4k-2}}}$ is
		\hfill \brak{JEE\ Adv.\ 2015}

	\item Let $\omega \ne 1$ be a cube root of unity. Then the minimum of the set $\cbrak{\abs{a + b\omega + c\omega^2}^2: a,b,c 
		\text{ distinct non-zero integers}}$ equals \rule{1cm}{0.15mm}.
		\hfill \brak{JEE\ Adv.\ 2019}
\end{enumerate}

\end{document}
