\let\negmedspace\undefined
\let\negthickspace\undefined
\documentclass[journal]{IEEEtran}
\usepackage[a5paper, margin=10mm, onecolumn]{geometry}
%\usepackage{lmodern} % Ensure lmodern is loaded for pdflatex
\usepackage{tfrupee} % Include tfrupee package

\setlength{\headheight}{1cm} % Set the height of the header box
\setlength{\headsep}{0mm}     % Set the distance between the header box and the top of the text

\usepackage{gvv-book}
\usepackage{gvv}
\usepackage{cite}
\usepackage{amsmath,amssymb,amsfonts,amsthm}
\usepackage{algorithmic}
\usepackage{graphicx}
\usepackage{textcomp}
\usepackage{xcolor}
\usepackage{txfonts}
\usepackage{listings}
\usepackage{enumitem}
\usepackage{mathtools}
\usepackage{gensymb}
\usepackage{comment}
\usepackage[breaklinks=true]{hyperref}
\usepackage{tkz-euclide} 
\usepackage{listings}
% \usepackage{gvv}                                        
\def\inputGnumericTable{}                                 
\usepackage[latin1]{inputenc}                                
\usepackage{color}                                            
\usepackage{array}                                            
\usepackage{longtable}                                       
\usepackage{calc}                                             
\usepackage{multirow}                                         
\usepackage{hhline}                                           
\usepackage{ifthen}                                           
\usepackage{lscape}
\begin{document}

\bibliographystyle{IEEEtran}
\vspace{3cm}

\title{3.3.7}
\author{AI24BTECH11031 - Shivram S
}
% \maketitle
% \newpage
% \bigskip
{\let\newpage\relax\maketitle}

\renewcommand{\thefigure}{\theenumi}
\renewcommand{\thetable}{\theenumi}
\setlength{\intextsep}{10pt} % Space between text and floats


\numberwithin{equation}{enumi}
\numberwithin{figure}{enumi}
\renewcommand{\thetable}{\theenumi}


\textbf{Question: }\\
Write the steps of construction for drawing a $\Delta ABC$ in which $BC = 8cm$,
$\angle B = 45 \degree$ and $\angle C = 30 \degree$.

\textbf{Solution: } \\

Mark points $\vec{B} = \myvec{0 \\ 0}$ and $\vec{C} = \myvec{8 \\ 0}$.

We know that the coordinates of $\vec{A}$ are

\begin{align}
    \vec{A} = \myvec{c\cos B \\ c\sin B} = \myvec{8 - b\cos C \\ b \sin C}
\end{align}

So,

\begin{align}
    \myvec{\cos C & \cos B \\ -\sin C & \sin B} \myvec{b \\ c} = \myvec{8 \\ 0}
\end{align}


By performing row reduction,

\begin{align}
    \augvec{2}{1}{
        \frac{\sqrt{3}}{2} & \frac{1}{\sqrt{2}} & 8 \\
        -\frac{1}{2} & \frac{1}{\sqrt{2}} & 0
    } \longleftrightarrow
    \augvec{2}{1}{
        1 & 0 & \frac{16}{\sqrt{3} + 1} \\
        0 & 1 & \frac{8\sqrt{2}}{\sqrt{3} + 1}
    } \\
    b = \frac{16}{\sqrt{3} + 1}, c = \frac{8\sqrt{2}}{\sqrt{3} + 1}
\end{align}

Hence,

\begin{align}
    \vec{A} = \myvec{c\cos B \\ c\sin B} = \myvec{\frac{8}{\sqrt{3} + 1} \\ \frac{8}{\sqrt{3} + 1}}
\end{align}

$\Delta ABC$ is the required triangle.

\begin{figure}[h!]
    \centering
    \includegraphics[width=0.7\linewidth]{figs/fig.pdf}
    \caption{Triangle $ABC$ where $BC=8cm$, $\angle B = 45 \degree$ and $\angle C = 30 \degree$}
\end{figure}

\end{document}



